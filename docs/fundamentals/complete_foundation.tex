\documentclass[11pt]{article}
\usepackage[utf8]{inputenc}
\usepackage{amsmath, amsfonts, amssymb, amsthm}
\usepackage{geometry}
\usepackage{graphicx}
\usepackage{hyperref}
\usepackage{cite}
\usepackage{booktabs}
\usepackage{array}

\geometry{margin=1in}

% Theorem environments
\newtheorem{theorem}{Theorem}[section]
\newtheorem{lemma}[theorem]{Lemma}
\newtheorem{corollary}[theorem]{Corollary}
\newtheorem{definition}[theorem]{Definition}
\newtheorem{proposition}[theorem]{Proposition}

\theoremstyle{remark}
\newtheorem{remark}[theorem]{Remark}

\title{On the Fundamental Oscillatory Nature of Physical Systems: A Mathematical Framework for Unified Dynamics}

\author{Kundai Farai Sachikonye}

\date{\today}

\begin{document}

\maketitle

\begin{abstract}
We present a mathematical framework establishing that oscillatory behavior represents the fundamental substrate of physical reality rather than an emergent property of complex systems. Through rigorous analysis of quantum mechanical wavefunctions, classical dynamical systems, and thermodynamic processes, we demonstrate that all physical phenomena can be described as manifestations of nested oscillatory hierarchies. We derive a generalized Lagrangian formulation that unifies quantum and classical mechanics through oscillatory coherence optimization principles. The framework provides mathematical resolution to the quantum-classical boundary problem and offers a coherent description of physical systems across all scales from quantum to cosmological.
\end{abstract}

\section{Introduction}

The ubiquity of oscillatory phenomena in physical systems has been recognized since the earliest developments of wave mechanics and harmonic analysis. However, conventional treatments regard oscillatory behavior as either a mathematical convenience or an emergent property of underlying particle dynamics. This work advances the hypothesis that oscillatory dynamics represents the fundamental substrate of physical reality, with apparent particle-like behavior emerging as a limiting case of coherent oscillatory patterns.

We establish this thesis through three primary mathematical developments: (1) demonstration that quantum mechanical wavefunctions are intrinsically oscillatory entities rather than probability amplitudes describing particle motion, (2) proof that classical dynamical systems with bounded phase spaces necessarily exhibit oscillatory behavior, and (3) derivation of a generalized Lagrangian mechanics based on oscillatory coherence optimization rather than energy extremization.
